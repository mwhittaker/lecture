% I use a custom document class that can be found at
% github.com/mwhittaker/texmf. Honestly, it's a big pain in the butt to set it
% up. Sorry this document isn't easier to compile! If you have the hw document
% class set up, you can compile this document with `latexmk -pdf answers.tex`

\documentclass{hw}
\title{CS 5220 -- 2015-09-08 Preclass Questions}

\hypersetup{
  colorlinks = true,
  allcolors = blue,
}

\begin{document}
\maketitle{}

\begin{enumerate}
  \item
    For all $i,j$, computing $C[i, j]$ requires reading $2N$ double precision
    numbers each of which is 8 bytes. This is a total of $16N$ bytes.  If $16N$
    is much bigger than the size of the cache, then every time we read in a
    value, we must bring it into cache from memory. We perform a single flop
    for every pair of numbers, so this leads to a flop per 16 bytes, or an AI
    of $\frac{1}{16}$.

  \item
    The matrix multiplication algorithm iterates over rows of $A$ and then
    iterates over columns of $B$. If our cache is larger than $16N$, we can
    cache the rows of $A$. Since the cache is smaller than $8N^2$, we won't be
    able to cache the columns in $B$. We still perform a flop per pair of
    values, but now one of these values is cached. This leads to an AI of
    $\frac{1}{8}$.

  \item
    Now we can read in all $16N^2$ bytes of $A$ and $B$ and perform $N^3$
    operations. Then, we write out $C$ which requires an additional $16N^2$
    memory writes for an AI of $\frac{N^3}{32N^2} = \frac{N}{32}$.

  \item
    I'll assume we're finding the biggest $N$ such that the caches can fit all
    of $A$, $B$, and $C$. The largest $N$ for  32 KB, 256 KB, and 6 MB are
    $105$, $296$, and $1449$ respectively. This leads to arithmetic intensity
    of $3.28$, $9.25$, and $35.9$ respectively.

  \item
    Your CPU can perform $4 \times 8 \times 2.4 = 78.6$ Gflops/s. This means
    that an arithmetic intensity of $\frac{78.6}{25.6} = 3$ will make the
    computation CPU bound.

  \item
    Matrix multiplication will be CPU bound for $N > 96 = 3 * 32$.

  \item
    Flops/second will increase with $N$ until $N$ becomes 96, at which point,
    the CPU becomes saturated and Flops/second will plateau.
\end{enumerate}
\end{document}
