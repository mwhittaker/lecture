% I use a custom document class that can be found at
% github.com/mwhittaker/texmf. Honestly, it's a big pain in the butt to set it
% up. Sorry this document isn't easier to compile! If you have the hw document
% class set up, you can compile this document with `latexmk -pdf answers.tex`

\documentclass{hw}
\title{CS 5220 -- 2015-09-17 Preclass Questions}

\hypersetup{
  colorlinks = true,
  allcolors = blue,
}

\begin{document}
\maketitle{}

\begin{enumerate}
  \setcounter{enumi}{-1}
  \item Between 1 to 2 hours on September 21.

  \item
    The slides assume a lot of familiarity with various mathematical concepts
    (e.g. ordinary differential equations, linear algebra, heat equation). It
    would be useful to review some of the mathematical prerequisites instead of
    assuming familiarity. Alternatively, I think a lot of the math could be
    removed entirely.

  \item
    See \texttt{csr\_product.c}.

  \item
    See \texttt{laplace2d.c}.

  \item
    Admittedly, I didn't fully understand the lecture slides nor do I full
    understand this question, but I'm guessing it involves some sort of
    ghost-celling in which a processor receives more cells that it needs in
    order to advance computation without communicating. I'm guessing there is
    no advantage to this scheme with only one processor.
\end{enumerate}
\end{document}
