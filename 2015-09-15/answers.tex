% I use a custom document class that can be found at
% github.com/mwhittaker/texmf. Honestly, it's a big pain in the butt to set it
% up. Sorry this document isn't easier to compile! If you have the hw document
% class set up, you can compile this document with `latexmk -pdf answers.tex`

\documentclass{hw}
\title{CS 5220 -- 2015-09-15 Preclass Questions}

\hypersetup{
  colorlinks = true,
  allcolors = blue,
}

\begin{document}
\maketitle{}

\begin{enumerate}
  \setcounter{enumi}{-1}
  \item
    I spent roughly an hour on the preclass assignment partly before class and
    partly after class.

  \item
    As a computer science major, I have very little exposure to differential
    equations of any variety. Since the lumped parameter and distributed
    parameter simulations are motivated by differential equations, it's very
    hard to understand what they are. Also, in the section on passing
    particles, we discuss that we can optimize code by having processors pass
    around memory. This is clear, but implementing such a thing seems unclear.

  \item
    Each Intel Xeon E5-2620 processor has 15MB of cache. The naive
    implementation of the game of life requires two $n \times n$ boards where
    each entry of a board is a single byte. This means we can fit a board of
    size $n = \sqrt{\frac{15000000}{2}} \approx 2738$.

    After running and timing a simulation with a $100 \times 100$ board for
    $1000$ generation and a simulation with a $4000 \times 4000$ board for
    $100$ generations, I found the former took $0.357$ seconds and the latter
    took $55.577$ seconds. This is a cell per second rate of $28011204$ and
    $2878887$ respectively.

  \item
    Assume we have an $n \times n$ board and want to partition the board into 4
    equal sized quadrants. Naively, each processor would get exactly
    $\frac{1}{4}$ of the board. However, in order to advance a single
    generation, the processor need to exchange the values of the cells on the
    boundary of the quadrants. Instead, we can assign overlapping quadrants
    where each processor gets slightly more than $\frac{1}{4}$ of the board.
    This would allow each processor to advance multiple generations without
    communication.

  \item
    In order to parallelize the code, each processor would get a partition of
    the $n \times n$ grid. They would locally update their grid and synchronize
    every once in a while. I think this would lead to considerable speedup on
    the totient cluster because the game of life is rather pleasingly parallel.

  \item
    Yes, for each particle in the processor's local memory, it must access
    every single particle. This requires it to fetch memory from all
    processors. It repeats this process for every particle which is very
    inefficient. It would be better to do a circular passing strategy, as
    discussed in the slides.
\end{enumerate}
\end{document}
