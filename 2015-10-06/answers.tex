% I use a custom document class that can be found at
% github.com/mwhittaker/texmf. Honestly, it's a big pain in the butt to set it
% up. Sorry this document isn't easier to compile! If you have the hw document
% class set up, you can compile this document with `latexmk -pdf answers.tex`

\documentclass{hw}
\title{CS 5220 -- 2015-10-06 Preclass Questions}

\hypersetup{
  colorlinks = true,
  allcolors = blue,
}

\begin{document}
\maketitle{}

\begin{enumerate}
  \setcounter{enumi}{-1}
  \item
    I spent roughly 2 hours on this assignment the day before lecture.

  \item
    The slides were pretty clear, though one thing I'm still confused about is
    what exactly is doing the message passing when using MPI? Do threads send
    messages, or is message passing at the level of a processor, a computer, or
    all of the above?

  \item
    \begin{enumerate}
      \item
        Running the ping-pong script using the Intel implementation of MPI, we
        get the output shown in \tabref{intel}.

        \begin{table}[h]
          \centering
          \begin{tabular}{|c|c|}
            \hline 1     & 6.6205e-07  \\\hline
            1001  & 1.26583e-06 \\\hline
            2001  & 1.60071e-06 \\\hline
            3001  & 8.9515e-07 \\\hline
            4001  & 1.08444e-06 \\\hline
            5001  & 1.24288e-06 \\\hline
            6001  & 1.42984e-06 \\\hline
            7001  & 1.6082e-06 \\\hline
            8001  & 1.78244e-06 \\\hline
            9001  & 1.94475e-06 \\\hline
            10001 & 2.13511e-06 \\\hline
            11001 & 2.3036e-06 \\\hline
            12001 & 2.4666e-06 \\\hline
            13001 & 2.61552e-06 \\\hline
            14001 & 2.7857e-06 \\\hline
            15001 & 2.95797e-06 \\\hline
            16001 & 3.19682e-06 \\\hline
          \end{tabular}
          \caption{Intel MPI}
          \label{tab:intel}
        \end{table}

      \item
        Running the ping-pong script using the Intel implementation of MPI, we
        get the output shown in \tabref{openmpi}. The performance is very
        similar to that of the Intel MPI library.

        \begin{table}[h]
          \centering
          \begin{tabular}{|c|c|}
            \hline
            1     & 4.86375e-07 \\\hline
            1001  & 1.25883e-06 \\\hline
            2001  & 1.67457e-06 \\\hline
            3001  & 1.00342e-06 \\\hline
            4001  & 1.02524e-06 \\\hline
            5001  & 2.01144e-06 \\\hline
            6001  & 2.28883e-06 \\\hline
            7001  & 2.31887e-06 \\\hline
            8001  & 2.52062e-06 \\\hline
            9001  & 2.6876e-06 \\\hline
            10001 & 2.8843e-06 \\\hline
            11001 & 3.03124e-06 \\\hline
            12001 & 3.09701e-06 \\\hline
            13001 & 3.2832e-06 \\\hline
            14001 & 3.41718e-06 \\\hline
            15001 & 3.68358e-06 \\\hline
            16001 & 3.81287e-06 \\\hline
          \end{tabular}
          \caption{OpenMPI}
          \label{tab:openmpi}
        \end{table}

      \item
        Say it takes $3 \times 10^{-6}$ seconds to send an MPI message. At
        a rate of 2.40 GHz, two nodes can complete $14,400$ floating point
        operations in the time it takes to send a single message.
    \end{enumerate}
\end{enumerate}
\end{document}
