% I use a custom document class that can be found at
% github.com/mwhittaker/texmf. Honestly, it's a big pain in the butt to set it
% up. Sorry this document isn't easier to compile! If you have the hw document
% class set up, you can compile this document with `latexmk -pdf answers.tex`

\documentclass{hw}
\title{CS 5220 -- 2015-10-08 Preclass Questions}

\hypersetup{
  colorlinks = true,
  allcolors = blue,
}

\begin{document}
\maketitle{}

\begin{enumerate}
  \setcounter{enumi}{-1}
  \item
    I spent one to two hours the day before class.

  \item
    The lecture slides and narration were pretty clear! I do have one
    suggestion though. The lecture slides introduced a lot of different
    functions, and there were plenty of pictures and code snippets to explain
    them, but it's hard to understand the operations without sitting down and
    experimenting with them. It would be nice to have a collection of minimal
    MPI programs that we could study, and a set of programming exercises to
    make sure we understand how to use them.

  \item
    I provided feedback in the CMS survey.

  \item
    \begin{enumerate}[(a)]
      \item
        \texttt{ring.c} involves a set of $n$ processes organized in a virtual
        ring topology. Each member of the ring has a local portion of some set
        of data. It computes locally, then sends its data to the next node in
        the ring while simultaneously receiving data from the node before it.
        This continues for a number of steps.

      \item
        Each thread copies the data it is handling into a third buffer. It then
        initiates an asynchronous send. While the messages are being sent, it
        computes local interactions. Once it's done, it waits to receive the
        next batch of data, copies the data, and repeats the process.
    \end{enumerate}
\end{enumerate}
\end{document}
