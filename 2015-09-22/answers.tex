% I use a custom document class that can be found at
% github.com/mwhittaker/texmf. Honestly, it's a big pain in the butt to set it
% up. Sorry this document isn't easier to compile! If you have the hw document
% class set up, you can compile this document with `latexmk -pdf answers.tex`

\documentclass{hw}
\title{CS 5220 -- 2015-09-22 Preclass Questions}

\hypersetup{
  colorlinks = true,
  allcolors = blue,
}

\begin{document}
\maketitle{}

\begin{enumerate}
  \setcounter{enumi}{-1}
  \item It took me $\approx 2$ hours to complete on September 21.

  \item
    When discussing the monte carlo simulation that computes the expected
    minimum distance between pairs of points, some of the details of expected
    value and variance were rushed. I think it would have been better to
    analyze the toy example that computes $\frac{\pi}{4}$ rather than tackle
    the more complicated example.

  \item
    \begin{enumerate}[a)]
      \item
        Each processor $p$ performs $\frac{N}{p}$ trials each of which takes
        $t_t$ time. Since the processors can perform the trials in parallel,
        all trials finish in $\frac{Nt_t}{p}$ time. Similarly, a total of
        $\frac{N}{b}$ batches must update global counters for a total of
        $\frac{N t_u}{b}$ time. This yields a total of
        \[
          \frac{N t_t}{p} + \frac{N t_u}{b}
        \]

      \item
        A run with $p = 1, b = 32$ takes $0.008289$ seconds across $1000032$
        trials. A run with $p = 32, b = 1$ takes $0.287262$ seconds across
        $1000030$ seconds. Substituting into formula above, we get
        \begin{align*}
          0.008289 &= \frac{1000032 t_t}{1} + \frac{1000032 t_u}{32} \\
          0.287262 &= \frac{1000030 t_t}{32} + \frac{1000030 t_u}{1}
        \end{align*}
        Solving these equations we find that $t_t = -6.88606 \times 10^{-10}$
        seconds and $t_u = 2.87275 \times 10^{-7}$ seconds.

      \item
        In this model, we want the batch size to be as large as possible
        because this minimizes the amount of global synchronization we need to
        do. In practice, I would sweep across values of $b$ and use the one
        that yields the smallest running time for a fixed $N$.
    \end{enumerate}

  \item
    See \texttt{workq.c}.
\end{enumerate}
\end{document}
